% \begin{table}[h]
% 	\centering
% 	\caption{\sys{}约束求解问题中的变量符号表示}
% 	\label{table:symbol}
%     \begin{tabularx}{\linewidth}{p{3cm} p{3.2cm} X }
%         \toprule
%         \textbf{输出模块} & \textbf{符号表示} & \textbf{描述} \\
%         \midrule
%         \multirow{2}*{显存估计} & $\mathit{weight}_u$ & 节点$u$的参数所使用的显存,其中$u\in V$ \\
%         \cmidrule{2-3}
%         & $\mathit{size}_e$ & 边$e$的起始节点的输出变量的大小,该变量作为边$e$的终点的输入,其中$e\in E$ \\
%         \midrule
%         \multirow{2}*{计算代价分析} & $\mathit{FP}_{u}$ & 计算图中节点$u$的进行前向传播的计算用时,$u\in V$\\
%         \cmidrule{2-3}
%         & $\mathit{BP}_{u}$ &  计算图中节点$u$的进行反向传播的计算用时,$u\in V$\\
%         \midrule
%         通信建模 & $\mathit{Commu}(x,p_1,p_2)$ & 表示设备$p_1$和设备$p_2$之间传输数据量$x$的通信用时,如果$p_1=p_2$,则通信代价可以忽略不计。$Commu$就是设备之间的通信代价模型。 \\
%         \bottomrule
%     \end{tabularx}
% \end{table}

\begin{table}[h]
	\centering
	\caption{\sys{}约束求解问题中的变量符号表示}
	\label{table:symbol2}
    \begin{tabularx}{\linewidth}{p{2.8cm}  X }
        \toprule
        \textbf{符号名称} & \textbf{描述} \\
        \midrule
        $\mathcal{S}_{fp,i}$  &在一次完整的前向传播和反向传播中,节点$i$进行前向传播的开始时间。\\
        \midrule
        $\mathcal{S}_{bp,i}$ & 在一次完整的前向传播和反向传播中,节点$i$进行反向传播的开始时间。 \\
        \midrule
        $\mathcal{C}_{fp,i}$ &在一次完整的前向传播和反向传播中,节点$i$前向传播的结束时间。\\
        \midrule
        $\mathcal{C}_{bp,i}$ &在一次完整的前向传播和反向传播中,节点$i$反向传播的结束时间。\\
        \midrule
        $\mathit{FP}_i$ & 节点$i$进行前向传播的用时,该用时由\textbf{计算代价估计}模块给出。\\
        \midrule
        $\mathit{BP}_i$ & 节点$i$进行反向传播的用时,该用时由\textbf{计算代价估计}模块给出。\\
        \midrule
        $F_{\mathit{commu}}(i,j,p,q)$ & 假设节点$i$被划分到设备$p$上,节点$j$ 被划分到设备$q$上,则$i$和$j$通过设备$p$和设备$q$之间到通信链路进行通信的用时为$\mathit{F_{\mathit{commu}}}(i,j,p,q)$。因为节点$i$和节点$j$之间通信的数据量是已知的,因此该用时可以由\textbf{通信代价建模}模块给出的$\mathit{Commu}(x,p,q)$进一步计算得到。\\
        \midrule
        $\mathcal{X}_{i,p}$ & 用于表示划分结果的指示变量,也是待求解的变量。$\mathcal{X}_{i,p}=1$ 表示节点$i$ 被划分到了设备$p$上。\\
        \midrule
        $\mathit{Cap_p}$ & 设备$p$的显存容量。在进行训练前,可以通过使用NVML库 \cite{nvml}采集硬件信息得到。\\
        \midrule
        $\mathit{Makespan}$ & 优化目标,模型训练中单次迭代用时。 \\
        \bottomrule
    \end{tabularx}
\end{table}