\begin{table}
	\centering
	\caption{ONNX与PyTorch的部分算子对应}
	\label{table:onnx-operators}
	\begin{tabularx}{\linewidth}{ p{2cm}  p{2.2cm}  p{5.8cm}  X  }
		\toprule
		\textbf{算子} & \textbf{输入} & \textbf{描述} & \textbf{对应PyTorch} \\
        \midrule
        Transpose & 
        input:tensor; perm:tuple; &
        根据perm对input的维度进行重新排列,功能类似于numpy中的np.permute
        & permute \\
        \midrule
        Flatten & 
        input:tensor; axis:int; &
        对输入张量input进行扁平化操作,输出output为2D矩阵。axis指定输出矩阵第一维的结束维度。
        例如输入张量的形状为$(a,b,c,d)$,axis=2,则输出矩阵的形状为$(ab, cd)$&
        flatten
        \\
        \midrule
        Upsample &
        input:tensor; scale:tensor;&
        对输入张量进行上采样,scale是一个长度等于input维数的张量,每一项指定了input的一个维度进行上采样时的缩放参数。例如对形状为$(a,b,c)$的输入张量进行上采样,scale=$(2,3,4)$,则输出张量的形状为$(2a,3b,4c)。$&
        interpolate
        \\
        \midrule
        Conv &
        input:tensor; w:tensor; bias:tensor; &
        使用w作为卷积核权重对输入张量input进行卷积运算,如果bias非空,则结果加上bias。&
        nn.Conv1d; nn.Conv2d; nn.Conv3d;
        \\
        \bottomrule
	\end{tabularx}
\end{table}