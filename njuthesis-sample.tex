%%%%%%%%%%%%%%%%%%%%%%%%%%%%%%%%%%%%%%%%%%%%%%%%%%%%%%%%%%%%%%%%%%%%%%
% njuthesis 示例模板 v1.2.1 2023-05-03
% https://github.com/nju-lug/NJUThesis
%
% 贡献者
% Yu XIONG @atxy-blip   Yichen ZHAO @FengChendian
% Song GAO @myandeg     Chang MA @glatavento
% Yilun SUN @HermitSun  Yinfeng LIN @linyinfeng
%
% 许可证
% LaTeX Project Public License(版本 1.3c 或更高)
%%%%%%%%%%%%%%%%%%%%%%%%%%%%%%%%%%%%%%%%%%%%%%%%%%%%%%%%%%%%%%%%%%%%%%

%---------------------------------------------------------------------
% 一些提升使用体验的小技巧:
%   1. 请务必使用 UTF-8 编码编写和保存本文档
%   2. 请务必使用 XeLaTeX 或 LuaLaTeX 引擎进行编译
%   3. 不保证接口稳定,写作前一定要留意版本号
%   4. 以百分号(%)开头的内容为注释,可以随意删除
%---------------------------------------------------------------------

%---------------------------------------------------------------------
% 请先阅读使用手册:
% http://mirrors.ctan.org/macros/unicodetex/latex/njuthesis/njuthesis.pdf
%---------------------------------------------------------------------

\documentclass[
    % 模板选项:
    %
    type = master, % 文档类型,默认为本科生
    % degree = academic|professional,        % 学位类型,默认为学术型
    %
    % nl-cover,   % 是否需要国家图书馆封面,默认关闭
    decl-page,  % 是否需要诚信承诺书或原创性声明,默认关闭
    %
    %   页面模式,详见手册说明
    % draft,                  % 开启草稿模式
    % anonymous,              % 开启盲审模式
    % minimal,                % 开启最小化模式
    %
    %   单双面模式,默认为适合印刷的双面模式
    % oneside,                % 单面模式,无空白页
    % twoside,                % 双面模式,每一章从奇数页开始
    %
    %   字体设置,不填写则自动调用系统预装字体
    latin-font = win,
    cjk-font   = win,
    % math-font  = cambria|newcm|xits, % 完整列表见手册
  ]{njuthesis}

% 模板选项设置,包括个人信息、外观样式等
% 较为冗长且一般不需要反复修改,我们把它放在单独的文件里
\input{njuthesis-setup.def}

% 自行载入所需宏包
% \usepackage{subcaption} % 嵌套小幅图像,比 subfig 和 subfigure 更新更好
% \usepackage{siunitx} % 标准单位符号
% \usepackage{physics} % 物理百宝箱
% \usepackage[version=4]{mhchem} % 绘制分子式
% \usepackage{listings} % 展示代码
\usepackage{algorithm,algorithmic} % 展示算法伪代码

% 在导言区随意定制所需命令
% \DeclareMathOperator{\spn}{span}
% \NewDocumentCommand\mathbi{m}{\textbf{\em #1}}

\usepackage{listings}
\usepackage{xcolor}
\usepackage{makecell}
\usepackage{threeparttable}
\usepackage{colortbl}
\usepackage{subcaption}
\usepackage{tabularx}
\usepackage{multirow}
\usepackage{pdfpages}
\captionsetup{compatibility=false}


\definecolor{codegreen}{rgb}{0,0.6,0}
\definecolor{codegray}{rgb}{0.5,0.5,0.5}
\definecolor{codepurple}{rgb}{0.58,0,0.82}
\definecolor{backcolour}{rgb}{0.95,0.95,0.92}

\renewcommand{\lstlistingname}{代码}
\lstdefinestyle{mystyle}{
    backgroundcolor=\color{backcolour},   
    commentstyle=\color{codegreen},
    keywordstyle=\color{magenta},
    numberstyle=\tiny\color{codegray},
    stringstyle=\color{codepurple},
    basicstyle=\ttfamily\footnotesize,
    breakatwhitespace=false,         
    breaklines=true,                 
    captionpos=b,                    
    keepspaces=true,                 
    numbers=left,                    
    numbersep=5pt,                  
    showspaces=false,                
    showstringspaces=false,
    showtabs=false,                  
    tabsize=2
}

\DeclareMathOperator*{\argmax}{arg\,max} % Jan Hlavacek
\DeclareMathOperator*{\argmin}{arg\,min} % Jan Hlavacek

\lstset{style=mystyle}

\def\sys{\textsc{NetSplit}}

%%%%%%%%%%%%%%%%%%%%%%%%%%%%%%%%%%%%%%%%%%%%%%%%%%%%%%%%%%%%%%%%%%%%%%%%%%%%%%%
% set up labelformat and labelsep for subfigure 详见: http://www.latexstudio.net/archives/8652.html
\captionsetup[subfigure]{labelformat=simple, labelsep=space}

% 开始编写论文
\begin{document}

%---------------------------------------------------------------------
%	封面、摘要、前言和目录
%---------------------------------------------------------------------

% 生成封面页
\maketitle

% 文档默认使用 \flushbottom,即底部平齐
% 效果更好,但可能出现 underfull \vbox 信息
% 如需抑制这些信息,可以反注释以下命令
% \raggedbottom

\begin{abstract}
  在深度学习领域,随着深度神经网络模型性能的提升,模型的复杂度和大小也在不断增加。单个计算设备已经无法满足大型神经网络模型的训练需求。因此,算法研究者通常使用模型并行化技术,将大型模型划分到多个计算设备上,进行分布式训练。
主流的深度学习框架中,对于模型的划分,仍然依赖使用者手动进行。由于模型结构复杂,加上底层设备的异构性,即使是对于有丰富经验的研究者,手动划分模型也是非常困难的任务。

现有的工作通过强化学习、启发式算法、构建约束优化问题等方法划分模型,但是目前的方法仍然存在一些不足。例如缺少对于底层硬件环境的考虑,对模型训练过程的建模不够精确等。
本文提出了一种针对大型深度神经网络模型的训练框架:\sys{}。具体而言,本文的主要工作包括:
\begin{itemize}
	\item 提出了一种自动化的PyTorch模型分析方法,可以从通用的PyTorch模型中提取出模型计算图的中间表示,并使用静态分析和动态分析的方法分析计算图中每个节点的计算时间和内存占用等元信息。
	\item 提出了一种对底层计算设备之间通信代价的建模方法:针对设备之间可能存在的异构通信链路,自动对设备之间进行点对点通信测试,并建模设备之间的通信代价。
	\item 提出了基于约束优化求解的模型划分方法:基于计算图元信息和设备之间的通信代价模型,构建约束优化问题并进行求解,对模型进行划分。
	\item 在真实场景下对\sys{}进行了实验评估,结果表明,和现有方法对比,\sys{}可以有效提升大型模型的训练效率,缩短训练时间。
\end{itemize}

\end{abstract}

\begin{abstract*}
  In the field of deep learning, with the improvement of the performance of deep neural network models, the complexity and size of the models are also increasing. A single computing device is no longer sufficient to meet the training requirements of large neural network models. Therefore, algorithm researchers usually use model parallelism to partition large models to multiple computing devices for distributed training.

In mainstream deep learning frameworks, the partitioning of models still relies on manual processing by users. Due to the complexity of model structure, coupled with the heterogeneity of underlying devices, even experienced researchers find it challenging to partition models manually.

Existing methods for partitioning models have used reinforcement learning, heuristic algorithms, and constrained optimization problems, among others. However, these methods still have some limitations, such as the lack of consideration for the underlying hardware environment and insufficient accuracy in modeling the model training process.

This article proposes a training framework \sys{} for large deep neural network models, which includes the following main work:

\begin{itemize}
    \item  Proposing an automated PyTorch model analysis method that can perform model structural analysis on generic PyTorch models, extract the intermediate representation of the model calculation graph, and analyze metadata such as calculation time and memory usage for each node in the calculation graph.
    \item  Proposing a modeling method for inter-device communication costs at the underlying computing devices for heterogeneous communication linkages, to facilitate automatic point-to-point communication testing between devices and modeling the communication costs between devices.
    \item Proposing a model partitioning method based on constrained optimization solutions for partitioning models based on the calculation graph metadata and inter-device communication costs. The method constructs an optimization problem and solves it for model partitioning.
    \item Conducting experimental evaluations of \sys{} in real-world scenarios which showed that \sys{} can effectively improve the training efficiency of large models and reduce training time compared to existing methods.
\end{itemize}
\end{abstract*}

% 生成目录
\tableofcontents
% 生成图片清单
\listoffigures
% 生成表格清单
\listoftables

%---------------------------------------------------------------------
%	正文部分
%---------------------------------------------------------------------
\mainmatter


% 学位论文的正文应以《绪论》作为第一章
\chapter{绪论}\label{chapter_introduction}
\section{研究背景}

近年来,人工智能领域的深度学习技术在各个应用领域都获得了迅速的发展。依赖于大数据时代数据量的爆发式增长,以及高性能计算设备的不断发展,深度神经网络(deep neural networks, DNN)在推荐系统 \upcite{recomendation1,recomendation2},图像识别 \upcite{resnet, imagenet, amoebanet}, 物体检测\upcite{yolo, yolov3} 以及自然语言处理 \upcite{bert,bart} 等应用领域都进行了广泛的应用,并且取得了巨大成功。

如图 \ref{fig:dnn-arch}所示,典型的深度神经网络由若干算子(卷积、池化、全连接等)搭建而成,不同的算子构建成层状结构,层与层被连接在一起,最终构成完整的神经网络模型。
深度神经网络主要依赖反向传播算法 (Back propagation) \upcite{bp}进行训练。以图像识别任务为例,在进行训练前,首先需要将模型的参数加载到内存中,然后针对训练数据,分批次迭代式的读取训练数据和数据对应的标签,首先经过前向传播,将数据输入到模型中,得到模型预测结果。然后将模型预测结果和真实标签输入到损失函数中,计算出损失,再进行反向传播,求解模型参数关于损失的梯度。得到梯度后,再根据优化器 \upcite{adam,rmsprop}的优化策略,对模型参数进行优化。重复上述过程,直到模型收敛。
从反向传播算法的运行过程可以看出,深度神经网络模型的训练是计算密集型任务,训练过程中需要涉及到高维数据的处理,反向传播求导,以及参数优化等。因此,目前的深度神经网络模型的训练通常需要使用专用的高性能计算设备来进行,例如GPU,TPU以及FPGA等 \upcite{tpu-gpu}。
\begin{figure}[h]
	\centering
	\includegraphics[width=0.8\textwidth]{figure/1-intro/dnn.png}
	\caption{典型深度神经网络结构}
	\label{fig:dnn-arch}
\end{figure}

以深度神经网络为代表的深度学习技术在性能提高的同时,也伴随着模型大小的增加和模型结构复杂度的提升。
以目前主流的高性能计算设备GPU为例,单个设备,无论是运算性能还是设备内存大小,都无法满足模型的训练需求,因此,使用多个设备进行训练,以提升训练效率,已经成为了深度学习领域的最佳实践。
如图 \ref{fig:model-mem} 所示,如果我们使用主流的ResNet-152模型,以$224\times 224$分辨率的输入,在32位浮点精度下,去训练图像识别任务,那么当数据批的大小达到$128$时,就会超过40GB的显存用量,此时,单个GPU设备的显存容量已经很难满足需求。

\begin{figure}[h]
    \centering
    \includegraphics[width=0.8\textwidth]{figure/1-intro/model_mem.pdf}
    \caption{主流计算机视觉模型显存消耗}
    \label{fig:model-mem}
\end{figure}

另一方面,数据集的大小也在不断增大。以ILSVRC-ImageNet2012 \upcite{imagenet} 为例,该数据集的训练集包括1000个分类类别,1300万张训练图片,训练集总大小超过120GB。如果使用单个设备进行训练,将所有图片训练一次(Epoch)需要超过数个小时的时间,而训练需要迭代式进行,往往需要经过多个轮次才能收敛,因此,使用多个计算设备加速训练过程是必要的。

综上所述,使用多个计算设备进行模型训练可以带来以下两点好处:
\begin{itemize}
	\item \textbf{允许训练更大的模型}: 受限于访存速度和制造成本,单个GPU设备的内存容量是有限的。针对大模型,模型的参数会占据大量内存空间,导致没有充足的内存空间去存储训练过程的中间结果、参数梯度等。而对于超大模型,甚至模型参数本身所使用的内存空间就会超过单个设备的内存容量。因此对于这类超大型深度神经网络模型(Giant DNN),使用多个设备进行协同训练是十分必要的。
	\item \textbf{提升训练效率}:模型训练通常使用批处理的方式进行训练,利用GPU的并行计算能力,同时处理一个数据批(Data batch)可以有效加速模型的训练速度,另一方面,使用数据批可以减少对模型参数进行梯度下降优化时的方差,加快收敛速度,现有工作 \upcite{large-batch} 表明,使用大数据批可以更好的代表样本总体,让模型准确的找到收敛方向。而使用大数据批的前提是需要有可用的显存空间,而使用多个GPU设备可以提供更多显存,从而允许算法研究者使用更大的数据批进行训练,提升训练效率,加快模型收敛速度。
\end{itemize}

随着深度学习的不断发展,使用多个设备加速深度神经网络的训练已经成为了最佳实践。
目前主流的开源机器学习框架,如PyTorch \upcite{pytorch} , Tensorflow\upcite{tensorflow} , MXNet\upcite{mxnet} 等,也提供了对多设备并行训练的支持,但是目前,对于超大模型的训练,主流框架多依赖于使用者对多设备协作训练进行手动配置,缺少对于超大模型训练的自动化支持。
尤其是在异构计算集群中,即使对于有经验的算法研究者,也无法总是准确考虑集群中的异构性,设备之间的通信特征以及模型的特点,这导致人工给出的模型训练配置往往不能很好的提高硬件利用率,达到最佳训练效率。因此,设计一种能够自动化完成超大模型端到端训练,自动进行模型划分,多设备协作的训练框架是十分有必要的。


\section{研究现状}
% 
目前,随着各种超大模型被设计出来,利用多设备去并行训练模型,加快模型训练速度的分布式深度学习 \upcite{ddl-survey}成为了机器学习领域的一个重要分支。
针对不同的场景,分布式深度学习有着不同的发展方向。目前,分布式深度学习有三种主要分支:
\begin{itemize}
	\item \textbf{数据并行化 (Data Parallelism, DP)}:数据并行化关注数据吞吐量的提升,也就是单位时间内,模型能够训练的数据样本数目。在数据并行化中,通常存在多个被称作Worker的计算单元,每个Worker使用一个设备,多个Worker之间进行协作完成训练。每个设备持有一个完整的模型副本,Worker使用不同的数据训练本地的模型,并按照一定的策略同步本地模型的参数,这就是数据并行化的思路。
	\item \textbf{模型并行化 (Model Parallelism, MP)}:针对超大模型在单个设备上训练受限的问题,模型并行化被提出,模型并行化也是本文研究的重点。与数据并行化不同,模型并行化针对如何将大型模型划分到多个设备上进行训练,每个设备只持有部分模型,反向传播算法在设备之间分布式运行,设备之间通过通信去传递中间结果以及反向传播过程中产生的梯度。在模型并行化中,由于每个设备只负责部分模型的训练,所以需要按照某些规则对模型进行划分,在划分时,需要考虑设备的负载均衡,避免某个设备的负载较大,成为瓶颈。另一方面,由于模型并行化中,设备之间需要进行大量通信,因此在划分时,也要考虑通信数据量、设备之间的通信链路对于训练效率的影响。
	\item \textbf{混合并行化 (Hybrid Parallelism, HP)}:混合并行化结合数据并行化和模型并行化二者的优势,先将超大模型划分为多个部分,每个部分由若干Worker构成的组进行训练,组内使用数据并行化,加大数据吞吐,组与组之间使用模型并行化,由于结合了二者的优点,因此混合并行化也是针对大模型使用的比较多的一种训练方式。
\end{itemize}

在模型并行化中,由于不同设备之间存在依赖关系,因此设备之间存在互相等待的现象,导致设备利用率低,针对这一问题,研究者专注于设计更好的通信模式,通过流水线的方式,让设备的计算过程和设备之间的通信过程重叠,从而提高设备利用率\upcite{gpipe,pipedream,hetpipe,hippie,dapple}。
Huang \upcite{gpipe} 等人提出了Gpipe技术,通过将数据批进一步细分成更小的微数据批(micro batch),将模型并行化以流水线的形式构建,加速模型并行化训练。
Narayanan \upcite{pipedream} 等人提出了PipeDream,在模型并行化流水线中引入异步计算,在内存中维护了多个版本的模型参数,以空间换时间,进一步提升训练效率。

除了使用流水线技术去改善模型并行化的训练效率外,模型并行化领域的另一个问题是,针对超大模型如何进行模型的划分 \upcite{rl1,rl2,baechi,pesto,rannc}。
Mirhoseini \upcite{rl1,rl2} 等人提出利用强化学习技术进行模型划分,但是随着模型大小的增加和模型复杂度的提升,强化学习收敛慢的问题在面对超大模型时变得更为突出。
Jeon \upcite{baechi} 等人将模型划分问题抽象为处理器调度问题,使用经典的任务调度算法,对模型进行划分。
相比于强化学习,尽管Jeon所使用的任务调度算法运行速度更快,但是其使用的算法是近似算法,在通信受限的场景下近似比较大,无法获得较优的结果。
Hafeez \upcite{pesto} 等人将模型划分问题进行建模,并利用线性规划进行求解。但是该工作只考虑了设备数目为2的简单情况,没有扩展到更多数目的设备,并且在进行问题建模时,没有考虑反向传播开销的影响。

综上所述,现有的模型划分方法仍然存在缺陷,使用模型并行化去训练超大模型时,如何去划分和放置模型仍然是一个亟待解决的问题。

\section{本文工作}
% 实现了xxx系统,使用xxx技术,解决了xxx问题
本文设计并实现了\sys,一种可以进行自动化模型划分与设备放置的超大型神经网络训练框架。
针对神经网络的模型划分问题,\sys 没有对神经网络进行粗粒度的分层切分,而是从神经网络底层的计算图出发,通过将原始模型转换成一种计算图的中间表示,以图的节点为单位对计算图进行细粒度的划分,并从划分后的计算图中重建出与原始模型等价的模型。
为了实现更好的划分效果,\sys 首先对训练使用的硬件环境进行性能测试,主要测试参与训练的设备之间的点对点通信带宽。
接着\sys 会对模型计算图进行分析,通过动态分析和静态分析的方法,获取计算图上每个节点的前向传播用时、反向传播用时、显存用量等模型元信息。
获取元信息之后,\sys 构建出模型划分与设备放置的约束优化问题,以提升训练效率为优化目标,进行求解。
求解完成后,\sys 将划分后的计算图按照求解结果进行设备放置,并基于PyTorch\upcite{pytorch} 实现后续的训练过程。

综上所述,本文提出了一种针对超大型神经网络模型的训练框架\sys,其主要的工作包括:
\begin{itemize}
	\item 基于开源机器学习框架PyTorch,实现了一种可以自动对通用PyTorch模型进行模型划分、设备放置以及训练的框架,可以有效解决大型神经网络在单个设备上难以训练的问题。
	\item \sys 可以根据训练任务所使用的硬件环境,自动化的对硬件进行点对点带宽测试,充分考虑模型并行化任务中,设备之间的通信链路对模型划分效果的影响。
	\item \sys 中实现了一种可以自动将通用的PyTorch模型转化成一种计算图中间表示的功能模块,基于这种计算图的中间表示,\sys 以计算图节点为粒度进行模型划分,相比于以层为粒度进行模型划分,更加细粒度,且更加通用。
	\item 设计了详细的对比实验:分别针对图像分类和图像分割两种任务,选用包括Wide-ResNet152\upcite{resnet} , AmoebaNetD-18\upcite{amoebanet}, U-Net\upcite{unet} 和DeepLab-V3\upcite{deeplabv3}在内的大模型,将\sys 与多种现有的模型划分与设备放置方法进行对比。实验结果显示,相比于现有方法,在不同模型上,\sys 的对于模型训练效率具有至少$4.4\% \sim 14.82\%$的提升。
\end{itemize}

\section{本文组织}
本文剩余章节组织如下:

第二章介绍与本文研究方向相关的技术背景和工作。
首先将介绍本文中出现的一些术语,然后会介绍主流的分布式深度学习技术与发展现状,模型并行化技术的发展,现有的模型划分技术,以及模型训练与计算图方面的工作。

第三章分析大型神经网络模型训练中的难点和挑战,并对问题进行概括,介绍了针对当前框架不足而设计的自动化的大型神经网络训练框架\sys{} 的设计。

第四章介绍\sys 各个功能模块的实现,包含模型转换模块、模型分析模块(显存/计算用时)、通信代价建模模块、计算图划分模块以及最后的模型训练模块。

第五章介绍\sys 中核心算法的实现细节,包括对问题的形式化描述、计算图优化算法、计算图划分算法等。

第六章对\sys 与其他模型划分与设备放置算法的实验进行详细介绍。包括实验对比方法、实验的硬件设置、实验的指标以及实验结果的分析。

第七章对本文工作进行总结,对当前工作中的不足进行分析,并展望本文工作未来的发展。
\chapter{背景知识与相关工作}

\section{模型、计算图与设备}
% 介绍一些专业术语
本节介绍一些与本文工作相关的术语和背景知识。
包括对神经网络模型、计算图以及神经网络训练设备的介绍。
\subsection{模型}
本文中所提到的模型指的是深度神经网络模型(Deep Neural Network, DNN)。
深度神经网络模型来自机器学习的分支领域深度学习,深度学习包括深度神经网络,深度信念网络,深度强化学习等。
深度神经网络模仿人脑中神经元的运作方式,通过激活函数连接模拟神经元。
多层感知机(Multilayer Perceptron, MLP) \upcite{mlp}是最简单的神经网络。
深度神经网络包含一个输入层,一个输出层和至少一层的隐藏层(Hidden Layer)。
每一层都由若干神经元组成,输入数据按照一定规则和神经元的参数进行运算,得到该神经元的输出。
输出经过非线性激活函数后,再作为下一层神经元的输入。
非线性激活函数的存在可以让深度神经网络以任意精度拟合任何函数,为复杂的非线性系统提供建模能力 \upcite{ml-book}。

深度神经网络的建模效果取决于神经元的参数,神经元的参数使用反向传播算法 \upcite{bp}进行训练更新。
反向传播算法的主要过程是首先通过前向传播获取网络的输出,然后选择损失函数(Loss Function),计算网络输出和真实标签的损失(Loss)。
然后反向传播,逐层求解神经元的参数关于损失的梯度,根据选用的优化方法进行梯度下降,更新参数,完成一次训练。
\begin{equation}
	\label{eq:bp}
	W_{i,j}(t+1) = W_{i,j}(t) + \eta \frac{\partial Loss}{\partial W_{i,j}(t)}
\end{equation}

公式 \ref{eq:bp}展示了反向传播算法中一次参数更新的过程。
其中的$Loss$ 代表损失函数计算出的损失,交叉熵损失函数 \upcite{cross-entropy}是一种常用的用于分类任务的损失函数。

\subsection{计算图}
% 深度学习框架的发展

为了帮助深度学习开发者更好的进行模型开发工作,越来越多的开源深度学习框架被设计出来,例如Google开源的Tensorflow \upcite{tensorflow},Facebook开源的PyTorch \upcite{pytorch}等。
随着深度学习框架多年的发展,从方便开发者的角度出发,框架提供更多高层的抽象封装,开发者可以使用高层的应用程序接口迅速搭建自己的模型,而不需要深入到模型底层,处理神经元之间的连接。
而神经元与神经元之间的连接所构成的图,就是深度学习中模型的计算图(Computational Graph)。

最初,开发者需要自己定义计算图中的节点,以及节点与节点之间的连接。
以Tensorflow的发展历程来看,在Tensorflow 1.x版本中,开发者需要手动定义计算图,因此开发一个复杂的模型的十分繁琐,而且静态的计算图给开发调试带来了很多不便。
而在Tensorflow 2.x版本中,则引入了动态图的设定,计算图隐藏在代码中,大大简化了模型的开发流程。
下面是在Tensorflow 1.x和2.x中定义 $a+b$ 的代码片段。

\begin{lstlisting}[language=Python, caption={静态图与动态图的比较}]
# tf 1.x: 手动定义计算图
import tensorflow as tf
a = tf.placeholder(tf.float32, name='var_a')
b = tf.placeholder(tf.float32, name='var_b')
add_op = tf.add(a, b, name='var_c')
sess = tf.InteractiveSession()
init = tf.global_variables_initializer()
sess.run(init)
res = sess.run(add_op, feed_dict={a:2., b:4.,})

# tf 2.x: 自动生成计算图
import tensorflow as tf
a = tf.constant(2.)
b = tf.constant(4.)
print('a+b=', a+b)
\end{lstlisting}


% 模型和计算图的关系
随着框架的发展,已经不需要静态定义完整的计算图,框架可以从开发者的模型定义中动态生成计算图。
在PyTorch和Tensorflow 2.x中,用户通过描述模型的计算过程来定义模型。
计算图则在模型训练时由框架从计算过程的定义中动态生成。
如图 \ref{fig:graph} 所示,这种动态生成计算图的方式帮助开发者屏蔽了底层的细节,提升开发效率,但是也为运行时模型划分带来了困扰。
计算图对用户来说是不透明的,所以从开发者层面无法控制计算图如何被放置,这也是本文工作尝试去解决的问题,
即针对动态图框架,自动的对计算图进行划分。

\begin{figure}[htbp]
	\centering
	\begin{minipage}[b]{0.66\textwidth}
	  \lstinputlisting[language=Python]{figure/2-background/graph.py}
	\end{minipage}
	\hfill
	\begin{minipage}[b]{0.3\textwidth}
	  \includegraphics[height=8cm]{./figure/2-background/graph.pdf}
	\end{minipage}
	\caption{模型定义和底层计算图}
	\label{fig:graph}
\end{figure}

\subsection{设备}
% GPU设备的特点
% % 和CPU的区别
% % 内存模型特点
% % 使用CUDA和CUDNN进行计算
% % GPU之间的连接方式介绍, 同个机器上 PCIe, NVLink, 跨节点使用NCCL
本文中所提到的设备特指GPU(Graphics Processing Unit),也就是图形处理器。
最初,GPU主要用于绘制图像和渲染图形,随着高性能计算需求的发展,GPU也发展出了其他的用途,例如用于深度学习中的模型训练、大规模并行处理、以及其他更为严苛的工作负载。
与CPU相比,GPU具有更多的处理单元。
CPU适用于通用的工作负载,特别是对于执行延迟敏感以及对单核性能要求更高的工作负载。
CPU将它数量相对较少的处理器核心集中用于处理单个任务,这让CPU更适合处理串行计算的任务。

通过单指令多数据技术(Single Instruction Multiple Data, SIMD),GPU中的多个计算单元可以同时在不同数据上执行同一条指令,这种架构让GPU适用于处理多维向量数据,同时也让GPU特别适用于进行并行计算。
图 \ref{fig:cpu-gpu}展示了CPU和GPU架构上的区别。
GPU对于高维向量的计算效率也受到了深度学习研究者的青睐,越来越多的深度学习负载运行在GPU上,使用GPU对神经网络和大量高维数据集进行深度学习训练已经成为了深度学习领域的最佳实践。
而GPU厂商也在积极为深度学习提供支持,由Nvidia® 开源的 CUDAToolkit \upcite{cuda}是目前最流行的GPU编程框架,可以让开发者使用GPU来加速自己的程序中的并行计算部分。

GPU通常需要处理大量高维数据,因此读取数据的带宽和数据加载的时延都具有较高的要求。
所以和CPU相比,GPU拥有自己独立的内存,通常称之为显存(Video Memory)。
以Nvidia® V100 GPU为例,它具有900GB/s的显存带宽,而通常的内存带宽只有20GB/s。
但是显存的容量也限制了设备能够运算的数据规模,顶级GPU通常有24~80GB的显存,但是与服务器内存相比,在面对大量数据时,仍然无法满足需求,当可用显存无法满足需求时,通常会引发内存溢出错误(Out of Memory, OOM)。
如何利用多个设备的组合来满足超大模型对的显存需求,也是本文的工作之一。

\begin{figure}[h]
	\centering
	\includegraphics[width=0.85\textwidth]{figure/2-background/cpu-gpu.png}
	\caption{CPU和GPU的架构}
	\label{fig:cpu-gpu}
\end{figure}

受限于单个设备的显存容量和计算能力,多个GPU也常在一起使用,进行并行计算和分布式计算。
在分布式计算过程中,GPU不仅需要处理自己的数据,通常也需要和其他GPU进行数据传输和交换。
使用传统的共享内存或者Socket的方式进行数据传输无法满足GPU之间通信速度的需求。
Nvidia® 开发的NCCL(Nvidia Collective Communication Library) \upcite{nccl}通信库提供了GPU之间进行通信的原语。
NCCL不仅提供了用于点对点通信的原语\texttt{send},\texttt{recv},还提供了例如\texttt{gather},\texttt{broadcast}等集合式通信原语,满足多节点的通信需求。

NCCL从软件层面提供了GPU设备之间通信的功能。在硬件层面,GPU之间有多种通信链路,如图 \ref{fig:gpu-commu} 所示,在计算集群中,GPU之间的通信链路可能有很多不同类型。
在同一个服务器的上的GPU之间可以通过PCIe swtich,也可以使用专有的GPU通信设备NVLink或者NV-Switch \upcite{nvlink}进行通信。
在跨节点通信方面,在非专用集群中,10~25Gbps Ethernet Network Interface Controller(NIC) 仍然是主流选择。
在对通信性能要求较高的场景中,使用 Ehternet NIC则无法满足需求,因此Remote Direct Memory Access(RDMA) \upcite{rdma} 技术被提出。
RDMA协议允许通信实体绕过操作系统内核而直接访问内存中的数据,可以大大提高通信带宽和降低通信延迟,使用RDMA协议的通信设备,例如 Infiniband \upcite{infiniband},可以达到100Gbps的通信带宽。

在分布式计算中,GPU与GPU之间异构的通信链路为系统的设计和工作负载的分发带来了很多挑战。
在分布式深度学习训练中,设备与设备之间需要传递大量的数据。
如果在进行工作负载分发时没有充分考虑集群中通信链路的异构性,则容易导致通信瓶颈的出现,继而影响整个任务的运行效率。
\begin{figure}[h]
	\centering
	\includegraphics[width=0.9\textwidth]{figure/2-background/commu.pdf}
	\caption{集群中GPU之间通信链路}
	\label{fig:gpu-commu}
\end{figure}

\section{分布式深度学习}
在大数据时代,随着新的应用场景的出现,有更多的数据被收集,而深度学习技术也被广泛的应用在分析数据,构建人工智能应用中, 例如自动驾驶系统 \upcite{self-driving1,self-driving2},AI智能编程助手Copilot \upcite{copilot}等。
目前,面对数据量的增加和模型复杂度的提升,算法研究者们面对的一个重要问题就是如何使用分布式的方式,加速深度神经网络的训练流程。另一方面,模型本身的容量也可能超过单个设备的限制,因此,需要考虑将模型划分到多个设备上,来缓解单个设备的容量不足的问题。
数据并行化(Data Parallelism) \upcite{dp} 与模型并行化(Model Parallelism) \upcite{mp} 分别是分布式深度学习的两种常用的模式,图 \ref{fig:dp-mp} 展示了这两种并行化模式的结构。
% 图 + caption说明
\begin{figure}
	\centering
	\caption{分布式深度学习的两种模式}
	\label{fig:dp-mp}
	\begin{subfigure}[b]{0.4\textwidth}
		\centering
		\includegraphics[width=0.85\textwidth]{figure/2-background/dp.pdf}
		\caption{数据并行化}
	\end{subfigure}
	\begin{subfigure}[b]{0.4\textwidth}
		\centering
		\includegraphics[width=0.85\textwidth]{figure/2-background/mp.pdf}
		\caption{模型并行化}
	\end{subfigure}
	\caption*{a. 数据并行化从训练数据的层面进行并行化,同时训练多个完整的模型副本,每个模型副本使用数据集的一个子集;b. 模型并行化从模型的层面进行并行化,模型并行化将单个模型划分到多个设备上,每个设备负责一部分模型的训练。}
\end{figure}

\subsection{数据并行化}
% 数据并行化的思想
% 数据并行化的两种经典结构 Ring和PS
尽管GPU等专用设备可以加速模型的训练,但是单个设备的算力和容量有限。
利用多个设备通过分布式的方式进行模型训练是加速训练过程的有效方式,多种分布式训练架构被设计出来。
为了应对海量训练数据,数据并行化技术被提出,数据并行化的主要思想为使用多个设备同时训练同一个模型的多个副本。
每个副本使用数据集的一个子集进行训练。
不同的副本周期性通过集合式通信(Collective Communication)进行同步。
在实践中,通常使用数据采样器,采样出一个较大的数据批,然后将数据批按照模型的副本数目进行等分,可以得到多个小数据批。
每个模型副本使用一个小数据批,使用反向传播算法进行训练,然后通过集合式通信,将所有的模型副本的参数梯度求和平均。
每个模型副本使用平均后的梯度来更新模型。

\begin{equation}
	\label{eq:dp}
	\begin{aligned}
		\frac{\partial Loss}{\partial W_{i,j}(t)} =& \frac{\partial \frac{1}{n} \sum_{k=1}^{n} f(x_k, y_k)}{\partial W_{i,j}(t)} = \frac{m_1}{n}\frac{\partial \frac{1}{m_1} \sum_{k=1}^{m_1} f(x_k, y_k)}{\partial W_{i,j}(t)} + \cdots + \frac{m_p}{n}\frac{\partial \frac{1}{m_p} \sum_{k=1}^{m_p} f(x_k, y_k)}{\partial W_{i,j}(t)} \\
		=& \frac{m_1}{n} \frac{\partial L_1}{\partial W_{i,j}(t)} + \cdots + \frac{m_p}{n} \frac{\partial L_p}{\partial W_{i,j}(t)} \\
		=& \frac{1}{p} \sum_{k=1}^{p} \frac{L_k}{\partial W_{i,j}(t)}
	\end{aligned}
\end{equation}

如公式 \ref{eq:dp}所示,
对于大小为$n$的数据批,将其分为$p$份,每份的大小为$m_i,\ 0\le i\le p-1$,每份数据都交给某个设备进行训练。
当满足$m_i=\frac{n}{p}$时,原始数据批关于参数的梯度等于每个小数据批关于参数的梯度的均值。
数据并行化中的关键步骤在于每次反向传播结束后,对不同模型副本的参数梯度进行同步。
参数梯度同步的目的是让模型副本之间共享彼此当前的训练的结果。
从参数梯度同步的架构上,主流的架构有参数服务器架构(Parameter Server, PS) \upcite{ps} 和All-Reduce \upcite{ring} 架构。

\begin{figure}
	\centering
	\begin{subfigure}[b]{0.48\textwidth}
		\centering
		\includegraphics[width=0.95\textwidth]{figure/2-background/ps.pdf}
		\caption{参数服务器架构}
		\label{fig:ps}
	\end{subfigure}
	\begin{subfigure}[b]{0.45\textwidth}
		\centering
		\includegraphics[width=0.95\textwidth]{figure/2-background/ring.pdf}
		\caption{Ring All-Reduce架构}
		\label{fig:ring}
	\end{subfigure}
	\caption{参数服务器架构 vs Ring All-Reduce架构}
	\label{fig:ps-ring}
\end{figure}

\begin{itemize}
	\item \textbf{参数服务器架构}: 参数服务器架构 \upcite{ps} 是最经典的分布式深度学习架构。如图 \ref{fig:ps} 所示,这种架构中,参与训练的节点被从逻辑上划分为了工作节点(Worker Node)和参数服务器节点(PS Node)。
	参数服务器节点的作用是存储模型的参数状态。
	工作节点的职责是进行模型的训练。
	当工作节点完成本地模型的训练后,会将本地最新的模型参数推送到参数服务器。
	参数服务器在接收到工作节点推送的模型参数后,会更新其存储到全局参数。
	然后工作节点再从参数服务器上拉取模型的全局参数。
	参数服务器可以有多个,每个参数服务器存储一部分模型。
	这种架构的优点在于工作节点和参数服务器节点的数量可以根据具体的硬件环境灵活调整,这让参数服务器架构拥有灵活的扩展能力。
	缺点在于参数服务器架构是中心化的架构,通信在参数服务器节点和工作节点之间频繁发生,而参数服务器节点往往需要同时和多个工作节点进行通信,容易导致通信瓶颈的产生。
	在参数服务器架构的基础上,衍生出多种改进版本。如BytePS \upcite{byteps},ElasticPS \upcite{elasticps},ParameterHub \upcite{pshub}等。

	\item \textbf{All-Reduce架构}: 以Ring All-Reduce \upcite{ring} 为代表的All-Reduce架构是一种去中心化的参数同步架构。
	如图\ref{fig:ring} 所示,在All-Reduce架构中,参与训练的计算节点被组织成为一个逻辑环。
	环上的每个节点都有唯一的前驱和后继,通信只发生在节点和前驱/后继之间。
	由于环上所有节点都是处于对称地位,所以每个节点的通信量是相同的,这也避免了参数服务器架构中的通信瓶颈的出现。
	每个节点在训练完成后,都将本地的模型参数分块后,在环上传递给自己的后继。
	在节点数目为$n$时,经过$2(n-1)$ 次通信就可以完成参数的同步。

	\item \textbf{Peer-to-Peer 架构}: 相比于对架构有严格限制的参数服务器架构和All-Reduce架构,Peer-to-Peer 架构采取了一种完全分布式的方式。每个工作节点都有完整的模型副本,工作节点之间直接进行点对点通信。
	这种方式具有更高的可扩展性,而且可以很好的处理分布式系统中的单点故障问题。
	受到这种思想的启发,Gossip Learning \upcite{gossip-learning} 被提出,并且用于分布式深度学习。
	Gossip Learning按照点对点通信网络中进行独立随机游走的想法构建。
	每个结点更新本地参数时,按照随机游走的方式,随机选取一组相邻节点,并将相邻节点的模型参数合并到本地。

\end{itemize}

从参数梯度同步的策略上,可以分为批同步方式(Bulk Synchronous Parallel, BSP),异步方式(Asynchronous Parallel, ASP)以及延迟同步方式(Stale Synchronous Parallel, SSP)。
\begin{itemize}
	\item \textbf{批同步策略(BSP)}: BSP策略 \upcite{bsp}是最简单的一种同步策略。
	使用BSP参数同步策略可以有效保证参与同步的参数的版本的一致性。
	经典的大数据并行计算架构MapReduce \upcite{mr}就使用了BSP同步策略。
	BSP通过对各个参与模型训练的节点的本地计算的进度进行同步来保证一致性,
	在每次迭代中,参与训练的工作节点首先读取训练数据,执行本地训练,当所有工作节点完成本地训练后,才进行接下来的参数同步。
	BSP的优点在于提供了对模型训练收敛性的保证,但是缺点是引入的同步障(Synchronization Barrier)容易造成工作节点之间的相互等待,运行速度快的节点需要等待运行速度慢的节点完成,一定程度上会影响训练的效率 \upcite{projadam}。
	BSP策略既可以用于参数服务器架构,也可以用于All-Reduce架构。
	\item \textbf{异步同步策略(ASP)}: ASP策略允许参与同步的节点不需要进行互相等待而可以并行通信。由于ASP中没有同步障,不存在相互等待,所以ASP策略下,单次同步非常快。
	但是ASP策略的缺点是模型可能永远不会收敛,在ASP同步策略中,参与同步的模型参数版本的差距可以是任意大,这将导致模型收敛慢或者无法收敛。
	\item \textbf{延迟同步策略(SSP)}: SSP策略是 \upcite{ssp}是对BSP和ASP的一种折衷。
	SSP允许参与同步的模型的参数版本的差距在一定范围内,该范围通过staleness 参数控制。
	如果参与同步的模型的参数版本的差距超过了限制,则退化为BSP,使用旧版本的模型参数进行同步。
	SSP的优点是缓解了BSP中的节点互相等待的情况,运行较快的节点不需要总是等待其他节点。
	SSP的缺点在于虽然SSP保证模型的收敛性,但是当staleness较大时,仍然可能影响收敛的效率。

\end{itemize}

\subsection{模型并行化}
% 模型并行化思想
数据并行化从数据角度进行并行,通过多路数据流进行训练加速。
而模型并行化从模型的角度进行划分,主要解决单个设备的内存有限,无法容纳大型模型的所有参数的问题。
如图 \ref{fig:dp-mp} 所示,在模型并行化中,模型被划分到多个设备上,每个设备负责部分模型的训练。
设备与设备之间通过相互通信传递前向传播的输入和反向传播的梯度,当训练完成后,所有设备上的模型组合在一起,得到完整的模型。
流行的深度学习框架如Tensorflow,PyTorch等,对数据并行化具有较为完善的支持,但是对模型并行化的支持有限,因为模型并行化中涉及到如何将模型划分到多个设备上,目前在PyTorch 1.13版本中引入了模型并行化的功能 \upcite{pytorch-pipeline},但是仍然依赖用户手动对模型进行划分。

% 介绍Gpipe等流水线并行化技术
在模型并行化研究领域,如何提高设备的计算利用率是另一个研究重点,由于模型并行化的特点,不同的设备之间有相互依赖。
设备之间通过通信传递数据,而设备的计算和通信过程无法重叠,当设备训练完本地的部分模型后,会进入通信过程,等待其他设备接受输出,以及等待来自其他设备的输入,如图 \ref{fig:mp-time} 所示,模型并行化中存在着设备利用率低下的问题。

\begin{figure}[h]
	\centering
	\begin{subfigure}[b]{0.85\textwidth}
		\centering
		\includegraphics[width=0.85\textwidth]{figure/2-background/mp-time.png}
		\caption{模型并行化}
		\label{fig:mp-time}
	\end{subfigure}
	\begin{subfigure}[b]{0.95\textwidth}
		\centering
		\includegraphics[width=0.85\textwidth]{figure/2-background/gpipe.png}
		\caption{流水线优化}
		\label{fig:gpipe}
	\end{subfigure}
	\caption{模型并行化 vs 流水线并行化}
	\label{fig:mp-gpipe}
\end{figure}

为了缓解模型并行化中,各个设备之间由于相互依赖导致的设备利用率低的问题,Huang等人提出了Gpipe \upcite{gpipe}。
Gpipe使用流水线技术优化模型并行化的过程,如图 \ref{fig:gpipe}所示,对于每个数据批,Gpipe将其进一步细分为微数据批(Micro Batch)。
某个设备处理完当前的微数据批后,将输出发送给下一个设备,然后立刻开始下一个微数据批的处理。
反向传播的过程也是类似,直到一个完整的数据批被处理完成,才更新所有的模型参数。
同时,Gpipe也提出了利用“重计算”优化显存使用。
设备不再暂存当前数据的输出结果,而是在进行反向传播计算梯度时,再重新执行前向传播过程,获取输出,再计算梯度。

除了Gpipe,使用流水线优化模型并行化的工作不断涌现。
Narayanan 等提出了PipeDream \upcite{pipedream}, 一种异步的流水线策略。
在PipeDream中,每个设备上同时维护多个模型参数的版本,当数据批到达时,选用最新的模型参数版本进行训练。
并且在之后的反向传播过程中,也使用相同版本的参数,以保持参数一致性。
Fan等人提出了Dapple \upcite{dapple}。Dapple组合了数据并行化和模型并行化。
模型被划分到多个设备组上,设备组之间使用模型并行化,设备组内部使用数据并行化。
此外Dapple还使用Early Backward技术来节省显存,通过调度每个设备进行前向传播和反向传播的顺序,优先让设备进行当前数据批的反向传播,释放掉中间结果的显存。

本小节中介绍了模型并行化以及在模型并行化上进行优化的一些相关工作。
这些工作主要从模型并行化的计算模式上进行优化,和本文工作\sys{} 中涉及的模型划分是正交的,属于不同的优化方向。
在\ref{sec:partition} 节中,将会介绍有关模型并行化中,进行模型划分的一些相关工作。


\section{模型划分技术}
\label{sec:partition}
% 介绍模型划分技术
% 维度: layer-level和operator level
% 比较,说明operator level粒度细,效果好
% 方向:
% 1. RL 
% 2. 经典算法
% 3. 约束优化
% 4. 启发式

在Gpipe等利用流水线对模型并行化进行优化的工作中,将模型定义为若干连续的层,对模型划分时也是进行以层为单位(Layer-wise)的划分。
尽管按层划分是一种可行的方案,但是其仍有如下的不足:
\begin{itemize}
	\item 限制了使用者定义模型的方式:按层划分只接受可以表示为连续若干层的模型,例如在PyTorch中,这类模型通常定义为\texttt{nn.Sequential},每一层只有一个前驱和一个后继。
	在进行划分时,连续的若干层会被划分到一个设备上。
	然而在实际进行模型开发时,模型并不总是可以表示为连续的若干层,例如一些结构比较复杂的模型,在PyTorch中通常会被定义为 \texttt{nn.Module},而这类模型并不支持按层划分。
	\item 按层划分难以达到最优的负载均衡:在模型并行化训练中,由于参与训练的设备之间存在依赖关系,因此需要谨慎的划分模型,让不同设备具有近似的负载,来避免负载不均。而层的定义也依赖于开发者,这容易导致层与层之间的大小(参数量、运算量)区别很大,例如在 基于BERT \upcite{bert} 的模型中,模型的最后一层占据了整个模型40\%的计算时间,对于这种巨大的层,还需要进一步的划分,才能保证不同设备的负载均衡。
	\item 按层划分是一种粗粒度划分方式:即使模型可以被表示为若干连续的层,那么对这些层进行划分的前提是层数大于设备数目,因为每个设备只少要被分到一层模型,但实际情况不总是如此。
\end{itemize}

在运行时,模型会被执行框架编译为计算图,再由运行时进行执行。
直接对计算图进行划分,也就是按照计算节点划分(Operator-wise) 可以有效避免按层划分的缺点。
从计算图层面对模型进行划分的工作有如下的方向:
\begin{itemize}
	\item \textbf{基于强化学习的方法}:Mirhoseini等提出了基于强化学习的计算图划分方法 \upcite{rl1,rl2}。
	Mirhoseini 使用的模型是NLP领域常用的Seq2Seq \upcite{seq2seq} 模型,并在该模型中加入了注意力机制。
	Seq2Seq模型由编码器(Encoder)和解码器(Decoder)两部分组成,编码器和解码器都使用LSTM(Long short-term memory) \upcite{lstm} 实现。
	该方法首先将计算图中的每个节点都进行编码,编码内容包括节点的类型,节点的输出大小,节点的邻接表。然后将节点编码序列输入到模型中,由模型输出每个节点的放置结果,也就是节点被放置到的设备编号。
	使用强化学习方法的缺点是需要很长的训练时间。
	\item \textbf{基于任务调度的近似算法}:Jeon等提出了使用基于任务调度的近似算法进行计算图的划分 \upcite{baechi}。
	计算图中的每个节点可以看作具有不同执行时长的任务,设备可以看作运行任务的处理器。
	节点之间的边则定义了任务的依赖,因此,从任务调度的角度,可以将计算图的划分问题建模为在有限个处理器上调度多个具有不同运行时长且具有依赖关系的任务。
	Jeon等使用经典的任务调度近似算法 \upcite{sched} 进行计算图的划分,包括最短任务优先(Earliest Task First, ETF) 和最短通信时间(Small Communication Time, SCT) 算法。
	\item \textbf{基于约束优化的方法}:计算图的划分问题在理论上被证明是NP-hard问题 \upcite{sched} ,因此无法找到多项式时间内的确定性算法对其进行求解,找到给定计算图和给定设备下的最优计算图划分。
	尽管Jeon等提出的近似算法可以快速得到近似解,但是由于算法近似比的限制,往往近似解无法取得和最优解一样的好结果。
	为了能够获得更好的划分结果,Hafeez等 \upcite{pesto} 将计算图的划分问题建模为一组约束优化,并使用高性能约束优化器进行求解。
	实验证明,该方法相比于强化学习和近似算法,可以得到更好的划分结果。

\end{itemize}

\section{本章小结}

本章首先对本文中出现的若干术语和概念,如模型、计算图和设备进行了详细的介绍。
然后介绍了分布式深度学习的发展和目前的两种主要模式,包括数据并行化和模型并行化。
有关数据并行化,介绍了其适用场景和主流架构,包括参数服务器架构和All Reduce架构,并对两种架构进行了对比。
有关模型并行化,介绍了模型并行化和数据并行化的区别,以及适用场景,并介绍了对于模型并行化的流水线优化技术。
最后,介绍了模型并行化中一个重要问题,也就是模型的划分问题。比较了分层划分和计算图划分两种方式,并介绍了目前对于模型划分的工作方向。

下一章讲介绍对大型神经网络模型进行训练的难点与挑战,以及我们针对大型神经网络模型训练提出的\sys{}框架的设计。 
\include{section/3-implementation}
\include{section/4-alg}

\chapter{实验评估}
\label{sec:evaluation}

本章将介绍对\sys{}所进行的实验评估。
首先我们介绍实验设置(\ref{sec:setup}),包括所使用的硬件环境以及选取的模型和数据集。
本章的研究旨在研究 \sys{} 的性能,验证\sys{}中提出的模型划分方法的有效性和稳定性。
具体而言,我们验证如下的两个问题:
\begin{itemize}
	\item \textbf{\textit{RQ1}}: 相比于其他现有方法,\sys{}能否有效提升模型的训练效率。
	\item \textbf{\textit{RQ2}}: \sys{}中提出的模型划分方法是否会影响原始模型的收敛性。
\end{itemize}

对于\textbf{\textit{RQ1}}, 我们在真实环境下选取了来自不同领域的深度神经网络模型,进行了模型并行化训练。
我们关注模型在不同的划分方法下的训练速度,训练速度越快,模型划分的效果越好(\ref{sec:performace})。
对于\textbf{\textit{RQ2}}, 我们设计了收敛性验证实验(\ref{sec:convergence}), 验证了\sys{}中提出的模型划分方法不会影响原始模型的收敛性。
最后我们对本章内容进行小结(\ref{sec:evaluation-summary})。

\section{实验设置}
\label{sec:setup}
\subsection{硬件环境}
\label{sec:hardware}

我们在一个服务器上进行实验,该服务器的配置如表\ref{table:setup}所示。
在该服务器上有5个型号相同GPU,每个GPU有24GB的显存,由于服务器上并未安装高性能的专用通信设备(例如NVLink)进行连接,因此GPU设备之间的通信链路是异构的,具体来说,GPU的通信拓扑如表\ref{table:topo}所示。

\begin{table}[!htbp]
	\centering
	\caption{服务器配置}
	\label{table:setup}
    \begin{tabularx}{0.9\linewidth}{ p{2.0cm} p{6.8cm} X}
        \toprule
        \textbf{配置项} & \textbf{型号} & \textbf{规格} \\
        \midrule
        GPU & Nvidia® TITAN RTX 24GB & 5 \\
        \midrule
        CPU & Intel® Xeon Gold 5118 @ 2.30GHz & 2 \\
        \midrule
        内存 & Samsung M393A4K40BB2-CTD  & 32GB $\times$ 12 \\
        \midrule
        操作系统 & Ubuntu & 4.15.0-88-generic \\
        \midrule
        \multirow{3}*{软件依赖} & PyTorch & 1.8.1 \\
        \cmidrule{2-3}
        & PICOS & 2.4.11 \\
        \cmidrule{2-3}
        & Gurobi & v10.0.0rc2 \\
        \bottomrule
    \end{tabularx}
\end{table}

\begin{table}[h!] % just use this specifier if really needed.
    \centering
    \caption{设备之间的通信拓扑}\label{table:topo}
    \begin{threeparttable}
    \begin{tabular}{ |p{1.5cm}| p{1.5cm}| p{1.5cm}| p{1.5cm}| p{1.5cm}| p{1.5cm}| }
        \hline
        & \textbf{GPU-0} & \textbf{GPU-1}& \textbf{GPU-2}& \textbf{GPU-3}& \textbf{GPU-4} \\
        \hline
        \textbf{GPU-0} & \texttt{X} & \texttt{PIX} & \texttt{NODE}&  \texttt{NODE}& \texttt{NODE} \\
        \hline
        \textbf{GPU-1} & \texttt{PIX} & \texttt{X} & \texttt{NODE} & \texttt{NODE} & \texttt{NODE} \\
        \hline
        \textbf{GPU-2} & \texttt{NODE} & \texttt{NODE} & \texttt{X} & \texttt{PIX} & \texttt{PIX} \\
        \hline
        \textbf{GPU-3} & \texttt{NODE}& \texttt{NODE} & \texttt{PIX} & \texttt{X} & \texttt{PIX} \\
        \hline
        \textbf{GPU-4} & \texttt{NODE}& \texttt{NODE} & \texttt{PIX}& \texttt{PIX} & \texttt{X}\\
        \hline
    \end{tabular}
    \begin{tablenotes}
        \item[1] \texttt{X}: 设备本身。
        \item[2] \texttt{PIX}: 设备之间通过单个PCIe switch连接。
        \item[3] \texttt{NODE}: 设备之间通过PCIe Host Bridge 在同一个NUMA节点内连接。
    \end{tablenotes}
    \end{threeparttable}
\end{table}

\begin{figure}[h]
	\centering
	\includegraphics[width=0.65\textwidth]{./figure/5-evaluation/pix-vs-node.pdf}
	\caption{PCIe Switch和PCIe Bridge的通信速度对比}
	\label{fig:switch-vs-bridge}
\end{figure}

PCIe(Peripheral Component Interconect Express) 是一种高速串行接口总线标准,用于将各种外部设备(例如显卡,网卡,硬盘等)连接到计算机上。
在服务器上,设备之间连接的方式主要有\texttt{PIX}和\texttt{NODE}两种。
\texttt{PIX}是设备通过PCIe Switch使多个PCIe设备连接在一起。在这种方式下,设备共享同一条PCIe总线,设备与设备之间具有高带宽和高吞吐量的连接。
而在\texttt{NODE}连接方式中,设备所使用的PCIe总线通过PCIe Bridge设备连接到NUMA(Non-Uniform Memory Access)节点内的PCIe总线。由于设备之间不共享PCIe总线,所以设备之间的带宽和吞吐相比于\texttt{PIX}有所降低,但是这种方式由于不共享总线,所以具有更好的可扩展性。
图\ref{fig:switch-vs-bridge} 展示了在我们的实验环境中,\texttt{PIX}和\texttt{NODE}两种连接方式下,设备之间通信的速度对比。
图中横轴表示通信数据量,纵轴表示通信时间。
从图中可以看出,通过\texttt{PIX}方式连接的设备之间具有更高的通信速度,反映出了在真实环境中,设备之间采用不同通信链路进行通信时的速度差距。


\subsection{模型与数据集}
在实验中,我们需要比较不同的模型划分方法对于模型训练速度的影响。
为了说明\sys{}中的模型划分方法具有通用性,我们分别从图像分类领域和图像分割领域选择数据集和模型进行实验。

\noindent\textbf{数据集}:在数据集方面,我们选择来自图像分类领域的ImageNet数据集\upcite{imagenet} 和来自图像分割领域的VOC2012数据集\upcite{voc2012}。
\begin{itemize}
	\item ImageNet: 该数据集包含了来自1000个类别的超过120万张高分辨率图像,用于训练和评估图像分类模型。数据集中的每个图像都被进行了标记。ImageNet在图像分类竞赛和图像分类的研究论文中作为基准数据集被广泛使用,因此我们选取ImageNet作为我们的实验数据集之一。
	\item VOC2012: VOC2012是用于目标检测和图像分割领域的数据集,包含来自11500个来自20个不同类别的对象的图像,如人、汽车、动物等。图像中的每个对象都带有一个指定对象在图像中范围的边界框和一个指示对象类别的标签。VOC2012被广泛应用为目标检测和图像分割领域的基准数据集,我们选取VOC2012作为我们的实验数据集之一。
\end{itemize}

\noindent\textbf{模型}: 在模型方面,我们在图像分类模型和图像分割模型中各选择两个大模型进行实验。在图像分类模型中,我们选择AmoebaNet-D\upcite{amoebanet} 和 Wide ResNet-152\upcite{wide}。在图像分割模型中,我们选择了U-Net\upcite{unet} 和DeepLab-V3\upcite{deeplabv3}。
\begin{itemize}
	\item AmoebaNet-D: AmoebaNet-D是AmoebaNet卷积神经网络架构的一种。它使用神经结构搜索算法(Neural Architecture Search, NAS)\upcite{nas} 设计,该算法可以根据给定任务自动搜索合适的网络结构。AmoebaNet-D的特点是具有大量的卷积层和复杂的模型结构,这使其可以学习到复杂的特征,因此AmoebaNet-D在图像分类任务上取得了很好的性能。
	\item Wide ResNet-152: 残差网络(Residual Network)是一种使用残差连接的深度神经网络,ResNet-152\upcite{resnet} 是残差网络的一种,它有152层,包括卷积、池化和全连接的组合。在相关研究\upcite{wide}中,研究者将ResNet中的卷积算子的通道数扩大后发现模型的性能得到了提高。我们在实验中将通道数ResNet-152的通道数扩大2倍,作为我们的待测模型。
	\item U-Net: U-Net是一种用于图像分割任务的卷积神经网络架构,其架构中包括一个对图像进行下采样的编码器和一个对图像进行上采样的解码器,通过采样可以让U-Net捕捉到图像在不同层次下的特征,目前U-Net已经被广泛应用于图像分割任务中。
	\item DeepLab-V3: DeepLab-V3是一种基于编码器/解码器结构的图像分割模型,通常,DeepLab-V3中,使用ResNet等其他卷积神经网络作为编码器,提取图像的特征,然后使用解码器从特征中生成分割结果。我们在实验中使用 Wide ResNet-152作为DeepLab-V3的编码器。
\end{itemize}
表\ref{table:model} 展示了我们所选用的模型在输入图片为$224\times 224$分辨率下,数据批大小为128时的内存需求,从表中可以看出,这些模型的内存需求已经远远超出我们的实验环境中的单个设备的内存容量(24GB)。

\begin{table}[h!] % just use this specifier if really needed.
    \centering
    \caption{模型内存需求}\label{table:model}
    \begin{tabularx}{0.86\linewidth}{ p{3.5cm} p{3.5cm} X  }
        \toprule 
        \textbf{模型名称} & \textbf{计算图节点数目} & \textbf{内存需求(MB)} \\
        \midrule 
        AmoebeNet-D & 784 & 117756.06 \\
        Wide ResNet-152 & 388 & 115179.77 \\
        U-Net & 134 & 50121.92 \\
        DeepLab-V3 & 417 & 168094.43 \\
        \bottomrule
    \end{tabularx}
\end{table}


\section{训练性能验证}
\label{sec:performace}
\subsection{对比方法}
为了验证\sys{}对于使用模型并行化对大模型进行训练的效果,我们引入了其他四种方法和\sys{}进行对比。

\begin{itemize}
	\item \texttt{m-TOPO}: \texttt{m-TOPO}是Baechi\upcite{baechi}提出的三种模型划分放置算法之一,全称为Memory-Constrained Topological Sort Placer。该算法的思路是对计算图中所有的节点进行拓扑排序,然后依次放置在当前设备上,直到达到当前设备的内存容量限制,再转向下一个设备继续放置。Baechi中将设备的内存容量限制设置为$\sum_{i\in\{n\}}d_i / m + \mathrm{max}_{i\in\{n\}} d_i$,其中$m$为设备数目,$d_i$为节点$i$的内存用量。通过这种方式,可以控制每个设备具有近似的内存用量。
	\item \texttt{m-ETF}: \texttt{m-ETF}是Baechi\upcite{baechi} 中提出的另一种模型划分放置算法,全称为Memory-Constrained Earliest Task First。该算法在最早任务优先调度算法\upcite{etf} 的基础上改进得到。该算法维护一个队列,队列中放置着所有的节点和设备的组合$(i,p)$,其中$i$表示节点,$p$表示设备。$(i,p)$按照最早可调度时间升序排列,最早可调度时间的计算按照式\ref{eq:etf}得到。
	\begin{equation}
		\label{eq:etf}
		\mathrm{ETF}(i,p) = \max\left[\mathit{free(p)}, \max_{i\in \Gamma^- (j)}(s_i+k_i+c_{ij}x_{ip}) \right]
	\end{equation}
	式\ref{eq:etf}中,节点$i$在设备$p$上的最早可调度时间为设备$p$的空闲时间和节点$i$的所有前驱节点$j$完成用时中的最大值。
	\texttt{m-ETF}依次从队列中取出当前可调度时间最小的组合$(i,p)$,如果设备$p$的剩余内存可以放置节点$i$,则将节点$i$放置到设备$p$,并更新设备$p$的空闲时间$\mathit{free}(p)$和$i$的所有后继的最早可调度时间。重复这个过程直到所有节点都被调度。

	\item \texttt{m-SCT}: 该算法是在经典的调度算法最小通信时间调度\upcite{sct}上改进得到。在\texttt{m-SCT}中,通过整数线性规划以最小化整体完成时间为目标,为每个节点寻找一个最优子节点,并优先将节点和其最优子节点放置在同一个设备上。
	\item \texttt{Pesto}: 同时,我们也对比了\texttt{Pesto}和我们的方法,由\texttt{Pesto}的原始约束优化问题(\ref{sec:pesto})只能适用于设备数目为2的情况,我们对其进行了扩展。我们采用和\sys{}类似的方法,使用多个指示变量来表示节点在设备上的放置。式\ref{eq:p-target-2-start} ~  \ref{eq:p-target-2-end}展示了修改后的\texttt{Pesto}的约束优化问题定义。
\end{itemize}

\begin{align}
	& \text{min} & & \mathit{Makespan} \label{eq:p-target-2-start} & \\
	& \text{s.t.} & & \mathcal{S}_{i}=0 &\Gamma^{-}(i)=0, i\in \mathcal{V} \\
	& & & \mathcal{C}_{i} = \mathcal{S}_{i} + \mathit{p}_i,\ \mathit{Makespan} \ge \mathcal{C}_{i} & \forall i\in \mathcal{V} \\
	& & & \sum_{p=1}^{m} \mathcal{X}_{i,p}=1, \mathcal{X}_{i,p}\in\{0,1\} & \forall i\in \mathcal{V} \\
	& & & \mathit{Commu}_{i,j} = \sum_{p\neq q}\mathcal{X}_{i,p}\mathcal{X}_{j,q}F_{\mathit{commu}}(i,j,p,q) & \forall \left\langle i,j\right\rangle \in \mathcal{E} \\
	& & & \mathcal{S}_{j} \ge \mathcal{C}_{i} + \mathit{Commu}_{i,j}&\forall \left\langle i,j\right\rangle \in \mathcal{E} \label{eq:p-target-2-end}
\end{align}






% \section{实验结果}
% 本节将介绍\sys{}和其他模型划分方法在性能上的对比(\ref{sec:performace})。此外,为了验证\sys{}并不会影响模型的收敛性,还将在\ref{sec:convergence}中对模型训练的收敛性进行验证,最后会在\ref{sec:partition-result}展示模型划分的效果图。

\subsection{实验结果及分析}


\begin{figure}[ht]
	\centering
	\begin{subfigure}[b]{0.45\textwidth}
	  \includegraphics[width=\textwidth]{./figure/5-evaluation/result-amoebanetd.pdf}
	  \caption{AmoebaNet-D}
	\end{subfigure}
	\quad
	\begin{subfigure}[b]{0.45\textwidth}
	  \includegraphics[width=\textwidth]{./figure/5-evaluation/result-deeplabv3.pdf}
	  \caption{DeepLab-V3}
	\end{subfigure}
	\vskip\baselineskip
	\begin{subfigure}[b]{0.45\textwidth}
	  \includegraphics[width=\textwidth]{./figure/5-evaluation/result-unet.pdf}
	  \caption{U-Net}
	\end{subfigure}
	\quad
	\begin{subfigure}[b]{0.45\textwidth}
	  \includegraphics[width=\textwidth]{./figure/5-evaluation/result-wide-resnet152.pdf}
	  \caption{Wide ResNet-152}
	\end{subfigure}
	\caption{\sys{}和其他模型划分方法在4个不同模型上的对比}
	\label{fig:result}
\end{figure}

我们选择\ref{sec:setup}中提到的模型和数据集进行训练,并在训练过程中采集100个数据批的训练用时,计算出每个数据批的平均用时和数据吞吐量。
我们选用3个GPU对每个模型进行训练,其中GPU-0和GPU-1以及GPU-2之间通过\texttt{NODE}的方式进行连接,GPU-1和GPU-2之间通过\texttt{PIX}的方式进行连接。
对于每一个模型,具体的配置为:
\begin{itemize}
	\item AmoebaNet-D: 我们使用18层的AmoebaNet-D,在ImageNet数据集上进行训练,数据批大小设置为64。
	\item Wide ResNet-152: 使用ImageNet数据集进行训练,数据批大小设置为64。
	\item U-Net: 使用VOC2012数据集进行训练,数据批大小设置为128。
	\item DeepLab-V3: 使用VOC2012数据集进行训练,数据批大小设置为48。
\end{itemize}

图\ref{fig:result} 展示了在相同的硬件环境下,在不同模型上使用不同的模型划分方法进行模型并行化训练时数据吞吐量的对比。
从图中可以看出,几种对比方法在不同的模型上表现各有优劣。而相比于几种对比方法,\sys{}在不同的模型上都获得了更高的数据吞吐量,可以在单位时间内完成更多的样本数据的训练,因此可以有效的提升训练速度。

\begin{table}
    \centering
    \caption{性能比较}\label{table:performance}
    \includegraphics[width=0.98\textwidth]{figure/5-evaluation/performance-table.pdf}
\end{table}

% \begin{table}[h!] % just use this specifier if really needed.
%     \centering
%     \caption{性能比较}\label{table:performance}
%     \tiny
%     \begin{tabular}{ |p{1.8cm} |p{2.0cm} |p{1.0cm} |p{1.3cm} |p{1.1cm} |p{1.1cm} |p{1.2cm} |p{1.4cm}| }
%         \hline
%         \multirow{2}{*}{\textbf{指标}} & \multirow{2}{*}{\textbf{模型}} & \multirow{2}{*}{\textbf{Pesto}} & \multicolumn{3}{c|}{\textbf{Baechi}} & \multirow{2}{*}{\textbf{NetSplit}}  & \multirow{2}{*}{\textbf{提升}}  \\ 
%         \cline{4-6} 
%         & & &\textbf{m-TOPO} & \textbf{m-ETF} & \textbf{m-SCT} & &\\  
%         \hline
%         \multirow{4}{*}{最大吞吐量} & AmoebaNet-D  &  34.74 & 37.14 & 19.87 & 29.05 & \textbf{42.73} & $\ge 15.04\%$ \\
%                                   \cline{2-8}
%                                   & W-ResNet152 &39.57 & 40.62 & 34.74 & 32.70 & \textbf{43.51} & $\ge 7.10\%$ \\
%                                   \cline{2-8}
%                                   & U-Net          &38.79 & 33.46 & 37.83 & 37.73 & \textbf{41.76} & $\ge 7.65\%$ \\
%                                   \cline{2-8}
%                                   & DeepLab-V3    &11.65 & 16.40 & 14.05 & 13.05 & \textbf{19.26} & $\ge 17.43\%$ \\
%         \hline
%         \multirow{4}{*}{最小吞吐量} & AmoebaNet-D  &  28.65 & 35.39 & 19.34 & 28.21 & \textbf{40.95} & $\ge 15.70\%$ \\
%                                   \cline{2-8}
%                                   & W-ResNet152 &37.27 & 38.82 & 33.28 & 31.53 & \textbf{41.25} & $\ge 6.27\%$ \\
%                                   \cline{2-8}
%                                   & U-Net          &38.60 & 33.30 & 37.62 & 37.53 & \textbf{39.72} & $\ge 2.90\%$ \\
%                                   \cline{2-8}
%                                   & DeepLab-V3    &11.40 & 15.90 & 13.92 & 12.02 & \textbf{17.30} & $\ge 8.81\%$ \\
%         \hline
%         \multirow{4}{*}{平均吞吐量} & AmoebaNet-D  & 30.17 & 36.46 & 19.59 & 28.82 & \textbf{41.83} & $\ge 14.72\%$ \\
%                                   \cline{2-8}
%                                   & W-ResNet152 &38.66 & 39.81 & 34.15 & 32.26 & \textbf{42.34} & $\ge 6.34\%$ \\
%                                   \cline{2-8}
%                                   & U-Net         &38.66 & 33.35 & 37.69 & 37.60 & \textbf{40.36} & $\ge 4.40\%$ \\
%                                   \cline{2-8}
%                                   & DeepLab-V3     &11.58 & 16.18 & 14.00 & 12.80 & \textbf{18.39} & $\ge 13.68\%$ \\
%         \hline
%         \multirow{4}{*}{\makecell{平均迭代用时\\(越低越好)}} & AmoebaNet-D    &2.13& 1.75& 3.26& 2.22&  \textbf{1.55} &  $\ge 11.43\%$\\
%                                                          \cline{2-8}
%                                                         & W-ResNet152 & 1.66& 1.59& 1.87& 1.98& \textbf{1.53} & $\ge 3.80\%$\\
%                                                          \cline{2-8}
%                                                         & U-Net          & 3.31& 3.84& 3.40& 3.40& \textbf{3.11} & $\ge 6.04\%$ \\
%                                                          \cline{2-8}
%                                                         & DeepLab-V3     & 4.15& 2.97& 3.43& 3.73& \textbf{2.79} & $\ge 6.06\%$\\
%         \hline
%     \end{tabular}
% \end{table}


% \begin{table}[h!] % just use this specifier if really needed.
%     \centering
%     \caption{性能比较}\label{table:performance}
%     \tiny
%     \begin{tabularx}{\linewidth}{ p{1.4cm} p{1.7cm} p{1.4cm} p{1.4cm} p{1.4cm} p{1.4cm} p{1.4cm} p{1.2cm}}
%         \toprule 
%         \multirow{2}{*}{\textbf{指标}} & \multirow{2}{*}{\textbf{模型}} & \multirow{2}{*}{\textbf{Pesto}} & \multicolumn{3}{c}{\textbf{Baechi}} & \multirow{2}{*}{\textbf{NetSplit}}  & \multirow{2}{*}{提升}\\ 
%         \cmidrule{4-6} 
%         & & &\textbf{m-TOPO} & \textbf{m-ETF} & \textbf{m-SCT} & \\  
%         \midrule
%         \multirow{4}{*}{最大吞吐量} & AmoebaNet-D    & & & & & & \\
%                                   \cmidrule{2-8}
%                                   & W-ResNet152 & & & & & & \\
%                                   \cmidrule{2-8}
%                                   & U-Net          & & & & & & \\
%                                   \cmidrule{2-8}
%                                   & DeepLab-V3     & & & & & & \\
%         \midrule
%         \multirow{4}{*}{最小吞吐量} & AmoebaNet-D    & & & & & & \\
%                                   \cmidrule{2-8}
%                                   & W-ResNet152 & & & & & & \\
%                                   \cmidrule{2-8}
%                                   & U-Net          & & & & & & \\
%                                   \cmidrule{2-8}
%                                   & DeepLab-V3     & & & & & & \\
%         \midrule
%         \multirow{4}{*}{平均吞吐量} & AmoebaNet-D    & & & & & & \\
%                                   \cmidrule{2-8}
%                                   & W-ResNet152 & & & & & & \\
%                                   \cmidrule{2-8}
%                                   & U-Net          & & & & & & \\
%                                   \cmidrule{2-8}
%                                   & DeepLab-V3     & & & & & & \\
%         \midrule
%         \multirow{4}{*}{平均迭代用时} & AmoebaNet-D    & & & & & & \\
%                                     \cmidrule{2-8}
%                                     & W-ResNet152 & & & & & & \\
%                                     \cmidrule{2-8}
%                                     & U-Net          & & & & & & \\
%                                     \cmidrule{2-8}
%                                     & DeepLab-V3     & & & & & & \\
%         \bottomrule
%     \end{tabularx}
% \end{table}

表\ref{table:performance}展示了几种不同的模型划分方式下,进行模型并行化训练时的性能比较。
对比表中的数据我们发现,\sys{}和其他的方法相比,在不同的模型和数据集上,可以有效提升平均数据吞吐量$4.4\% \sim 14.72\%$,相应的,平均迭代用时,也就是模型训练每个数据批的时间减少了$3.8\% \sim  11.43\%$。
由于采样误差和训练过程中其他开销(数据预处理,磁盘IO等)的影响,导致平均数据吞吐量的提升和平均迭代用时的减少略有一些差异。
总的来说,相比于其他方法,使用\sys{}进行模型划分,可以有效提升模型并行化的训练效率,提升集群中的硬件资源利用率,缩短训练任务的用时。


在几种被测模型中,对于模型大小更大的AmoebaNet-D和DeepLab-V3,\sys{}的表现最好,相比于其他几种方法,在吞吐量方面有超过$13.68\%$的提升。
而对于模型大小较小的U-Net,\sys{}相比于其他几种方法中表现最好的\texttt{Pesto}来说,提升只有$4.40\%$。
对于AmoeBaNet-D和DeepLab-V3这种计算图中含有大量节点和边的复杂模型,在进行训练时,设备需要进行大量运算来得到反向传播的中间结果。同时由于模型的计算图中有大量的边,导致模型训练过程的实际通信代价较大,因此针对异构通信环境并且在约束优化问题的求解中加入反向传播的\sys{}在这样的复杂模型上表现更好。而没有考虑反向传播用时和异构的通信环境的\texttt{Pesto}在面对这样的复杂模型时,性能反而不如一些近似算法。

\subsection{有效性威胁分析}
\label{sec:threat}
% 1. 我们在单个主机的多个设备上进行实验,在这种情况下,p2p通信只发生在设备内部。尽管我们没有将实验扩展到多个主机的多个设备,并且让部分通信发生在网络上,但是这并不会影响建模准确性。因为即使是在单个主机上,设备之间的通信链路也是异构的,从实验结果上,在单机上有效
% 2. 在数据集和模型的选择上,我们选择来自图像领域的模型和数据集,而没有选择如文字、音频和视频等数据集。why? 图像识别有代表性,不同的任务领域的训练模式和特征是一样的,都是迭代式训练。


在实验场景的选择上,在\sys 对于模型训练性能的提升的实验评估中,我们将模型划分到位于单个主机的多个设备上进行训练,此时设备之间的通信只发生在主机的内部。而在真实的模型训练场景下,模型有可能被划分到位于多个主机的多个设备上,这种情况下,部分通信通过主机之间互联的网络进行。因此在单机场景下的实验也许无法反映真实场景下不同方法的性能差距。但是在单个主机上,设备之间的通信链路也是异构的,例如设备之间有可能通过PCIe Switch 连接,也有可能通过PCIE Host Bridge 连接,这种异构性和真实场景下的通信类似,因此,\sys 中对于问题的建模可以用于真实的场景。未来可以尝试更加真实场景下的设备和通信方式。



在数据集和模型的选择上,我们选择了图像领域的模型和数据集作为实验对象,因此,我们尚不清楚我们的框架在其他领域的模型和数据集的效果如何。但由于图像识别是最具有代表性、最被广泛研究的深度学习任务之一,并且我们选择的模型和数据集都是广泛被使用的基准测试,我们相信我们的评估结果在一定程度上是可靠的。我们将在未来工作中增加对文本、音频、视频等领域模型和数据集的评估。


\section{训练收敛性验证}
\label{sec:convergence}
我们设计了训练收敛性实验来验证\sys{}中对模型所进行的转换操作以及模型划分操作不会影响原始模型的性能。

\subsection{对比方法}
\begin{table}[h!] % just use this specifier if really needed.
    \centering
    \caption{收敛性验证实验配置}\label{table:convergence-setup}
    \begin{tabularx}{0.66\linewidth}{ p{3.5cm} X  }
        \toprule 
        \textbf{配置项} & \textbf{配置值}\\
        \midrule 
        模型 & ResNet-152 \\
        数据集 & ImageNet \\
        数据批大小 & 64 \\ 
        设备数目 & 4 \\
        学习率 & 1e-4 \\ 
        优化器 & Adam Optimizer \\
        \bottomrule
    \end{tabularx}
\end{table}

如本文\ref{sec:convertion} 中所述,当用户向\sys{}提交以PyTorch中\texttt{nn.Module}格式定义的模型后,\sys{}会将用户的模型转化成计算图的中间表示,在划分完成后,\sys{}还会将中间表示转化成\texttt{fx.GraphModule}。
在进行第二步转化时,会在模型中根据划分结果插入一些额外的节点,完成跨设备的变量转移。
因此为了验证经过多次转化和修改后的模型是否还和原模型在功能上等价,我们设计了收敛性验证实验。

收敛性验证实验的目的是为了验证\sys{}并不会损害原始模型的性能,我们通过测试使用\sys{}训练转换后的模型和使用数据并行化(Data Parallelism, DP)训练的原始模型的收敛性进行验证。
对于数据并行化,我们使用PyTorch中提供的DistributedDataParallel(DDP)\footnote{\url{https://pytorch.org/docs/stable/notes/ddp.html}} 进行训练。
我们使用的实验配置如表\ref{table:convergence-setup} 所示。


\begin{figure}[ht]
	\centering
	\begin{subfigure}[b]{0.8\textwidth}
	  \includegraphics[width=\textwidth]{./figure/5-evaluation/convergence-epoch.pdf}
	  \caption{模型准确率随轮次(Epoch)变化}
	  \label{fig:convergence-epoch}
	\end{subfigure}
	\vskip\baselineskip
	\begin{subfigure}[b]{0.8\textwidth}
	  \includegraphics[width=\textwidth]{./figure/5-evaluation/convergence-time.pdf}
	  \caption{模型准确率随时间变化}
	  \label{fig:convergence-time}
	\end{subfigure}
	\caption{收敛性验证}
	\label{fig:convergence-all}
\end{figure}

\subsection{实验结果及分析}

图\ref{fig:convergence-all} 展示了模型收敛性的验证结果。
其中,图\ref{fig:convergence-epoch}展示了在两种不同的训练方式下,模型的准确率随着训练轮次(Epoch)变化的情况。
我们一共训练了40个轮次,从图中可以看出,在相同的轮次下,两种训练方式得到的模型的准确率几乎相同。
因此可以得到结论,\sys{}并不会损害原始模型的性能,即不会影响原始模型的收敛性。
此外,图\ref{fig:convergence-time} 展示了模型的准确率随着时间的变化情况。
可以看出,由于\sys{}利用了模型并行化方法,有效提升了大模型训练的效率,相比于数据并行化方法,每一轮所需的训练时间更短,模型随着时间的收敛速度更快。


\section{本章小结}
\label{sec:evaluation-summary}

在本章中,我们详细介绍了针对\sys{}进行的两方面实验,分别是训练性能验证和训练收敛性验证。
训练性能验证实验中,我们将\sys{}和其他的模型并行化方法在真实的模型和数据集上进行训练,通过比较训练时的数据吞吐量来评估训练性能。
实验结果表明,相比于已有方法,\sys{}可以有效提升对大模型进行模型并行化训练的训练效率,缩短训练任务用时。
在训练收敛性验证实验中,我们比较\sys{}和数据并行化的对在相同条件下对同一个模型的训练过程,实验结果表明\sys{}不会影响模型训练过程的收敛性。

下一章(\ref{sec:summary})将总结本文的工作,并展望未来的工作。
\include{section/6-summarization}

% 符号表
% 语法与 description 环境一致
% 两个可选参数依次为说明区域宽度、符号区域宽度
% 带星号的符号表(notation*)不会插入目录
% \begin{notation}[10cm]
%   \item[DFT] 密度泛函理论 (Density functional theory)
%   \item[DMRG] 密度矩阵重正化群 (Density-Matrix Reformation-Group)
% \end{notation}

% 建议将论文内容拆分为多个文件
% 即新建一个 chapters 文件夹
% 把每一章的内容单独放入一个 .tex 文件
% 然后在这里用 \include 导入,例如
%   \include{chapters/introduction}
%   \include{chapters/environments}


%---------------------------------------------------------------------
%	参考文献
%---------------------------------------------------------------------

% 生成参考文献页
\printbibliography

%---------------------------------------------------------------------
%	致谢
%---------------------------------------------------------------------

\begin{acknowledgement}
  回首过去,这已经是我在南京大学的第七个年头。在南京大学我留下了许多美好的回忆,在此,我要向在我求学生涯中,帮助过我的各位老师,各位同学表达我的感谢。

首先要感谢的是我的导师曹春老师以及徐经纬老师。
在三年求学过程中,你们的教诲和引导使我逐渐建立了自己的学术兴趣和研究方向,让我在求学过程中能够不断探索、学习、成长。
你们的关心和鼓励,让我在研究生求学过程中感受到了莫大的温暖和动力。你们的教诲将伴随着我未来的人生道路,让我受益终生。
同时感谢南京大学计算机系的各位老师,你们丰富多彩的课程让我获得很多宝贵的知识。

感谢实验室的所有同学,你们让我在求学路上不孤单,和你们共同学习,共同工作的每一天都让我受益良多。

最后感谢我的家人和我的女朋友朱庭纬同学,你们的陪伴和关心让我有勇气去面对困难,有信心去跨越坎坷。

此致,感谢!




\end{acknowledgement}

%---------------------------------------------------------------------
%	附录部分
%---------------------------------------------------------------------

% 附录部分使用单独的字母序号
% \appendix

% 可以在这里插入补充材料
% 完工

% \includepdf[pages=-]{section/resume.pdf}
\njuchapter{简历与科研成果}
\section*{基本信息}
\noindent 崔子寒,男,汉族,1998年1月出生,江苏徐州人。

\section*{教育背景}

\noindent \textbf{2020 年 9 月 — 2023 年 6 月}~南京大学计算机科学与技术系 \hfill 工学硕士

\noindent \textbf{2016 年 9 月 — 2020 年 6 月}~南京大学计算机科学与技术系 \hfill 理学学士

\section*{攻读硕士学位期间的成果}
\begin{enumerate}[label=\arabic*., labelindent=0em, leftmargin=*]
    \item 曹春、徐经纬、\textbf{崔子寒}:一种面向深度学习模型分布式训练的容器自动编排方法(专利
    申请号:CN202211426263X)。
\end{enumerate}

\section*{攻读硕士学位期间参与的科研课题}
\begin{enumerate}[label=\arabic*., labelindent=0em, leftmargin=*]
    \item 国家重点研发计划:软件定义的人机物融合云计算支撑技术与平台(2018YFB1004805)
    \item 广东省重点领域研发计划: 面向云数据中心智能管控的软件定义方法与关键技术 (2020B010164003)
    
\end{enumerate}


\end{document}
